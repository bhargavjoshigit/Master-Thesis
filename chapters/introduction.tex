\chapter{Introduction}


Industrial Automation in manufacturing refers to the use of “Intelligent” machines in the factories so that the manufacturing process can be carried out with as low a human interaction as possible. In spite of the detractors, the purpose of Industrial automation is not to force the workers out of their jobs, instead to let them focus on more complex jobs while the machines do the more monotonous, tiring and dangerous jobs. 


Industrial automation was a booming field in the early 1930s due to the high demands of manufactured goods and the not so high production capabilities of the companies. And so many companies set up new departments for the Research and Development of Automation technology. Ever since the dream of having fully automated factories has been growing progressively larger and complex, and we have come a long way since, where we now have huge factory set-ups operating at 90-95\% autonomy, with human intervention at a bare minimum.


Use of Augmented Reality (AR) for Industrial automation is an attempt to explore further the possibilities of having better automation for tasks that still need a lot of human interactions. AR is most notably associated with the gaming and entertainment industries, but it’s not only about fun and games, AR can and is being used in many other application such as Education, Medicine, Sports and Industrial automation among others.


In this Thesis I will look for the solutions to the question of how AR can be used in a Manufacturing industry, in order to help the user get a better output, in particular we will look at how AR can be used to help the User in setting up a workpiece in a CNC machine. We will look at “why and how”, AR “should and can” be used in Industries. We will also look at the state of art AR technology. We will then dive in to take a look at the actual application that I created for demonstrating the use of AR, where we will explore the software and hardware that is needed to develop such an application. We will ask and try to answer the questions about the feasibility of the AR technology and how easily adaptable it is in the current Manufacturing industry. We will then see the future outlook and the possible improvements that are needed and can be done.


